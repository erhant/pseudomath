
\documentclass{article}
\usepackage{listings} 
\usepackage{color} %use color
\usepackage{amsmath}
\usepackage{mathpazo}
\usepackage[mathpazo]{flexisym}
\usepackage{breqn} 

\newcommand*{\mysqrt}[4]{\sqrt[\leftroot{#1}\uproot{#2}#3]{#4}}

\lstset{ 
backgroundcolor=\color{white},
basicstyle=\footnotesize, 
breakatwhitespace=false, 
breaklines=true, 
captionpos=b,
commentstyle=\color{mygreen},
deletekeywords={...}, 
escapeinside={\%*}{*)}, 
extendedchars=true,
frame=single,
keepspaces=true,
keywordstyle=\color{blue},
morekeywords={*,...},
numbers=left,
numbersep=5pt,
numberstyle=\tiny\color{mygray},
rulecolor=\color{black},
showspaces=false,
showstringspaces=false,
showtabs=false,
stepnumber=1,
stringstyle=\color{mymauve}, 
tabsize=2,
title=\lstname 
}
 
\definecolor{darkgray}{rgb}{.4,.4,.4}
\definecolor{purple}{rgb}{0.65, 0.12, 0.82}
\definecolor{mygreen}{rgb}{0,0.6,0}
\definecolor{mygray}{rgb}{0.5,0.5,0.5}
\definecolor{mymauve}{rgb}{0.58,0,0.82}
 
\lstdefinelanguage{JavaScript}{
keywords={typeof, new, true, false, catch, function, return, null, catch, switch, var, if, for, in, while, do, else, case, break},
keywordstyle=\color{blue}\bfseries,
ndkeywords={class, export, boolean, throw, implements, import, this},
ndkeywordstyle=\color{darkgray}\bfseries,
identifierstyle=\color{black},
sensitive=false,
comment=[l]{//},
morecomment=[s]{/*}{*/},
commentstyle=\color{purple}\ttfamily,
stringstyle=\color{red}\ttfamily,
morestring=[b]',
morestring=[b]"
}
 
\lstset{
language=JavaScript,
extendedchars=true,
basicstyle=\footnotesize\ttfamily,
showstringspaces=false,
showspaces=false,
numbers=left,
numberstyle=\footnotesize,
numbersep=9pt,
tabsize=2,
breaklines=true,
showtabs=false,
captionpos=b
}
\begin{document}
\title{AUTOMATICALLY GENERATED LATEX}
\maketitle

\subsection{INPUT CODE}
\begin{lstlisting}[language=JavaScript]
addmul(a, b, c)
{
   t = a + b;
   t = t * c;
   return t;
}

\end{lstlisting}

\subsection{CONVERTED LINES}
\begin{equation*}\begin{split}
addmul \left(a,b,c \right) = L_{2} \left(a,b,c,t \right)
\end{split}\end{equation*}
\begin{equation*}\begin{split}
L_{2} \left(a,b,c,t \right) = L_{3} \left(a,b,c,t \right)
\end{split}\end{equation*}
\begin{equation*}\begin{split}
L_{3} \left(a,b,c,t \right) = L_{4} \left(a,b,c,a + b \right)
\end{split}\end{equation*}
\begin{equation*}\begin{split}
L_{4} \left(a,b,c,t \right) = L_{5} \left(a,b,c,t  \times  c \right)
\end{split}\end{equation*}
\begin{equation*}\begin{split}
L_{5} \left(a,b,c,t \right) = t
\end{split}\end{equation*}
\begin{equation*}\begin{split}
L_{6} \left(a,b,c,t \right) = \infty
\end{split}\end{equation*}


\subsection{SQUISHED LINES}
\begin{equation*}\begin{split}
addmul \left(a,b,c \right) = a + b  \times  c
\end{split}\end{equation*}


\subsection{CONVERTED CODE}
\begin{lstlisting}[language=JavaScript]
function addmul(a,b,c) {
    return a + b * c;
}

\end{lstlisting}

\end{document}
